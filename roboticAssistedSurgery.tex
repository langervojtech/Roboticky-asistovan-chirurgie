\documentclass[10pt, a4paper]{article}
\usepackage[utf8]{inputenc}
\usepackage{amsthm}
\usepackage{amsfonts}
\usepackage{amssymb}
\usepackage{amsmath}
\usepackage{latexsym}
\usepackage{graphicx}
\usepackage{blindtext}
\usepackage{ragged2e}
\usepackage{doc}
\usepackage{url}
\usepackage{natbib}

\newtheorem*{theorem}{Theorem}
\theoremstyle{definition}
\newtheorem*{definition}{Definition}

\hoffset -1in \topmargin 0mm \voffset 0mm \headheight 0mm
\headsep0mm
\oddsidemargin  20mm     %   Left margin on odd-numbered pages.
\evensidemargin 20mm     %   Left margin on even-numbered pages.
\textwidth   170mm       %   Width of text line.
\textheight  252mm

\makeatletter
\renewcommand\@openbib@code{%
     \advance\leftmargin  \z@ %\bibindent
      \itemindent \z@
     % Move bibitems close together
     \parsep -0.8ex
     }
\makeatother

\makeatletter
\renewcommand\section{\@startsection {section}{1}{\z@}%
                                   {-3.5ex \@plus -1ex \@minus -.2ex}%
                                   {1.5ex \@plus.2ex}%
                                   {\large\bfseries}}
\makeatother

\makeatletter
\renewcommand\subsection{\@startsection {subsection}{1}{\z@}%
                                   {-3.5ex \@plus -1ex \@minus -.2ex}%
                                   {1.5ex \@plus.2ex}%
                                   {\normalsize\bfseries}}
\makeatother

\makeatletter
	\setlength{\abovecaptionskip}{3pt}   % 0.25cm 
	\setlength{\belowcaptionskip}{3pt}   % 0.25cm 
\makeatother
%%%%%%%%%%%%%%%%%%%%%%%%%%%%%%%%%%%%%%%%%%%%%%%%%%%%%%%%%%%%

\begin{document}
\pagestyle{empty}

\begin{center}
{\bf \Large ROBOTICKY ASISTOVANÁ CHIRURGIE}
\end{center}

\smallskip
\begin{center}
{\large Vojtěch Langer}
\end{center}

\smallskip
\begin{center}
Faculty of Mechanical Engineering, Brno University of Technology\\
Institute of Automation and Computer Science\\
Technicka 2896/2, Brno 616 69, Czech Republic\\
Vojtech.Langer@vutbr.cz\\
\end{center}
%%%%%%%%%%%%%%%%%%%%%%%%%%%%%%%%%%%%%%%%%%%%%%%%%%%%%%%%%%%%

\bigskip
\noindent Abstract:
\textit{Tato zpráva pojednává o problematice roboticky asistované chirurgie.
Jsou uvedeny možnosti, přednosti a úskalí roboticky asistované chirurgie,
jací roboti jsou nejčastěji používáni a jaké druhy robotů a technologií
jsou vyvíjeny a představovany v současných odborných pracích.}

\vspace*{10pt}
\noindent Keywords:
\textit{minimally invasive surgery,
laparoscopic, robot assisted surgery,
surgical robotics, da Vinci}
\bigskip
%%%%%%%%%%%%%%%%%%%%%%%%%%%%%%%%%%%%%%%%%%%%%%%%%%%%%%%%%%%%

\section{Úvod}
\label{sec:uvod}

Roboticky asistovaná chirurgie je rychle rozvíjejícím se oborem.
Pomocí robotické chirurgie mohou lékaři provádět jemné a složité postupy,
které by u jiných metod mohly být velice obtížné.
Proto již v roce 1985 byl použit první robot pro přesné navádění nástrojů při operaci.
V současnosti se robotů využívá ve většině odvětví chirurgie a to především
při minimálně invazivních zákrocích. \cite{mis_thesis},\cite{mis_mou}

Minimally Invasive Surgery (MIS - minimálně invazivní chirurgie)
poskytuje stejnou ne-li vyšší kvalitu zákroku, s výrazně menším vstupem
oproti klasické otevřené operaci. Jedním z typů roboticky asistované MIS
je také laparoskopie - druh břišní chirurgie, při které se na rozdíl od otevřené operace,
provádějí úkony pod kontrolou kamery, bez přímého přístupu chirurga.
Ze zdravotního pohledu, laparoskopie vyžaduje značnou dovednost,
kvůli vnitřním technickým omezením, které mají za následek ztrátu
koordinace oko-ruka a sníženou obratnost. Proto je laparoskopie
jednou z disciplín moderní chirurgie, kde se s výhodou využívá robotů. \cite{minirobot}

Hlavní výhodou roboticky asistované laparoskopie je jednodušší
manipulace s nástroji než jaké může být dosaženo při klasické laparoskopii.
Učící křivka pro úplné naučení ovládání robota je přibližně 10 provedených robotických
zákroků, přičemž se operační doba sníží přibližně na polovinu. Hlavním omezením je velký průměr
nástrojů $8mm$ a omezené množství robotických rukou. Omezený počet
robotických rukou a nástrojů je pak obzvlášť poznatelný při zvládání krvácení. \cite{plusminus},\cite{robot_vs_video}

Pro pacienta je možnost MIS zásadní výhodou. Rány vzniklé při robotické operaci
jsou mnohem menší a odpadají problémy s jejich hojením. 
Rekonvalescence a návrat do běžného života jsou po robotickém zákroku velmi rychlé, 
zatímco po klasické operaci v nezanedbatelné míře hrozí komplikace, například výskyt kýly v jizvě, 
což si často vyžádá další chirurgický zákrok. 
Pacienti samozřejmě velmi pozitivně hodnotí také vynikající kosmetický efekt. \cite{mis_hom}

%%%%%%%%%%%%%%%%%%%%%%%%%%%%%%%%%%%%%%%%%%%%%%%%%%%%%%%%%%%

\section{Řešení robotů}
\label{sec:robot}

V této kapitole jsou prezentováni moderní roboti, kteří jsou využíváni v praxi
či byli představeni ve vědeckých pracích v posledních letech.
Prvním příkladem komerčního chirurgického robota, který byl ve větší míře
úspěšně aplikován v praxi byl robot da Vinci od společnosti Intuitive Surgical Inc.
Da Vinci je v současnosti nejrozšířenějším systémem pro použití v MIS.
Další roboti nejsou komerčně dostupní jako
systémy od firmy Intuitive nebo nejsou určeny pro klasickou chirurgii s nástroji.
Za zmínku stojí pak stojí AESOP (1994), ZEUS (1998) a Sofie (2012) - Surgeon's Operating Force-feedback Interface Eindhoven.\cite{wiki}

\subsection{Intuitive da Vinci}
\label{subsec:da_vinci}

Chirurgické systémy od firmy Intuitive Surgical Inc. byly představeny na trh v roce 2000
a od té doby se staly nejpoužívanějšími robotickými systémy pro MIS.
Název da Vinci si zasloužil podle Leonarda da Vinci a jeho zájmu o
studium lidské anatomie.
Hlavní výhodou chirurgických systémů od Intuitive 
je samotná velikost firmy, s čímž souvisí pak také lepší možnosti
technické podpory jak softwaru tak hardwaru, dostupnost náhradních dílů
a případně doplňků na specializované úkony.
Navzdory nesporným výhodám, systém Da Vinci je vcelku objemný, drahý a
náročný na zřízení v operační místnosti.
Systémy od Intuitive jsou používány po celém světě. V ČR
například ve Vojenské nemocnici v Praze a v univerzitních nemocnicích
v Olomouci a Brně. \cite{mis_thesis},\cite{intuitive},\cite{mis_kraj}

Systém da Vinci se skládá ze 3 hlavních podsystémů:
Edgonomicky tvarované konzole chirurga, vozíku Vision a vozíku pacienta.
Chirurg, který je usazen u konzole chirurga, řídí
pohyby nástrojů a endoskopu pomocí dvou ručních ovladačů
a sady nožních pedálů. Chirurg sleduje obraz z endoskopu na 2 LCD obrazovkách,
což mu poskytuje pohled na anatomii pacienta a na nástroje, společně s ikonami
a dalšími prvky uživatelského rozhraní.

Vozík pacienta je umístěn u operačního stolu a má 4 ramena, která jsou
umístěna s ohledem na cílovou anatomii pacienta.
Endoskop se připojuje na jedno z ramen. Samotný nosný systém má
několik stupňů volnosti přičemž ramena samotná mají podle modelu až 5 kloubů.

Vozík Vision obsahuje podpůrné elektronické zařízení, jako je zdroj světla
a zařízení zpracovávající video a obraz pro endoskop a hlavní elektronickou
a softwarovou výpočetní jednotku. Vozík je také vybaven
dotykovou obrazovkou na niž lze sledovat obraz z endoskopu a provádět
nastavení systému.
\cite{da_Vinci_handbook}


%%%%%%%%%%%%%%%%%%%%%%%%%%%%%%%%%%%%%%%%%%%%%%%%%%%%%%%%%%%
\subsection{Minirobot pro zatahovací úkony}
\label{subsec:minirobot}\label{}
V článku \textit{A miniature robot for retraction tasks under vision assistance in Minimally Invasive Surgery} \cite{minirobot}
je zkoumán nový přístup laparoskopické chirurgie.
V laparoskopii i jednoduché chirurgické úkony jako je vyjmutí orgánu
mohou být výzvou, protože jsou prováděny skrz malý vstupní otvor.
Z tohoto důvodu je neustálá potřeba vyvíjet nové robotické nástroje
pro zatahovací úkony.

Autoři článku navrhli malého robota o průměru $12mm$ a celkové délce $52mm$
včetně jeho magnetické základny. Navrhovaný robot má 2 stupně volnosti.
Je vybaven bezkartáčovým motorem v magnetické základně a v rameni, 
pro dosažení jednoduchých pohybů otáčení kolem osy a výklonu $180$\textdegree. 
Díky jeho rozměrům může být snadno naváděn k
požadovaného orgánu a aplikovat zpětnou výtažnou sílu až $1,53N$.
Robot byl navržen jako sériový kinematický řetězec,
jelikož pak může být jednoduše využito
chirurgického trokaru (trubicový nástroj pro inserci).
Byla navržena také varianta s kamerou $320\times240p$ $30fps$ místo drapáku.

Byl proveden experiment, při kterém otestovali možnosti tohoto robota.
Testovali maximální zdvižnou sílu a sílu drapáku, přičemž byly
pohyby monitorovány kamerou.
Pro vyhodnocení sil použili snímač síly s rozlišením $3,18mN$.
Maximální naměřená zdvižná síla byla $1,53N$ a maximální síla drapáku $5,3N$.
Hmotnost robota ($12g$) mírně ovlivňuje maximální zdvižnou sílu.

Následně byla simulována simulace pro demonstraci proveditelnosti
jednoduché procedury. Procedura se skládala z přivedení robotických 
komponent do břišní dutiny skrz jícen. Demonstraci provedli s úspěchem,
a proto doporučují robota k dalšímu testování na živých pacientech.

\subsection{Měkký robot}
\label{subsec:soft}
Studie \textit{Intelligent Soft Surgical Robots for Next-Generation Minimally Invasive Surgery} \cite{intelligent_soft_robot}
se zabývá možnostmi použití měkkých robotů v MIS.
Měkký robot se vyznačuje lepší přizpůsobivostí a bezpečnější
interakcí. Proto je měkká robotika označována za velmi slibnou při řešení 
současných výzev v MIS, které je obtížné řešit současnými prostředky.
V článku jsou analyzovány očekávané charakteristiky MIS nové generace
z pohledu chirurgů a je komplexně shrnut nedávný pokrok
měkkých chirurgických nástrojů ze tří různých aspektů: konstrukční
návrh, výrobní techniky a interakce člověk-robot.
Následně jsou diskutovány perspektivy měkkých chirurgických robotů
nové generace, kde jsou zdůrazněny některé zajímavé možnosti.

Autoři předpokládají, že další vývoj inteligentní měkké robotiky
umožní MIS nové generace obratnou a hbitou navigaci k cíli a provádět
diagnostické nebo terapeutické postupy bez jakýchkoli kompromisů
v invazivnosti a nakonec být vhodným řešením pro budoucí chirurgii.

\subsection{Robot s proměnlivou tuhostí}
\label{subsec:stiffness}
Léčba karcinomů žaludku nad $20mm$ v průměru může být dosažena
provedením endoskopické disekce pomocí flexibilního endoskopu(optický přístroj pro zobrazení vnitřních dutin).
Tento postup je však technicky náročný, zákrok vyžaduje delší
operační čas a rozsáhlé školení.

Pro usnadnění endoskopické disekce
autoři článku \textit{Deployable, Variable Stiffness, Cable Driven Robot for Minimally Invasive Surgery} \cite{variable_stiffness_robot}
vytvořili kabelem poháněného robota, který má zvýšit
možnosti flexibilního endoskopu a zároveň se snaží
minimalizovat dopady chirurgického zákroku.
Pomocí nízkoprofilové nafukovací nosné konstrukce ve tvaru
dutého šestihranu se robot může složit kolem flexibilního
endoskopu a po dosažení cílového místa dosáhnout $73,16\%$ zvýšení objemu
a jeho radiální tuhosti.
Plášť kolem struktury s proměnnou tuhostí poskytuje řadu kabelů pro přenos
síly, které se připojují ke dvěma nezávislým trubicovým koncovým efektorům,
na které mohou být ukotveny standartní flexibilní endoskopické nástroje.
Pomocí jednoduchého ovládacího schématu založeného na délce
každého kabelu lze polohu nástrojů ovládat haptickými ovladači v rukou chirurga.
V rámci studie byly změřeny síly vyvíjené jediným přístrojem a podél jedné
osy byla pozorována maximální síla $8,29N$. Toto sestavení bylo použito během zákroku
napodobující požadavky endoskopické disekce, kterou úspěšně provedl
začínající uživatel.

Autoři si od tohoto řešení slibují usnadnění náročné chirurgických zákroků
a možnost jednoduše přizpůsobit robota a následně ho zrychleně vyrobit
s nízkými náklady díky programovému přístupu návrhu.

\subsection{4-DOF Origami}
\label{subsec:4DOF}
V článku \textit{A Novel 4-DOF Origami Grasper With an SMA-Actuation System for Minimally Invasive Surgery} \cite{4DOF}
je prezentován 4-DOF origami drapák, pro
vývoj miniaturizovaného chirurgického nástroje.
s drapákem tvoří soustavu se čtyřmi stupni volnosti.
Práce v tomto článku zkoumá kinematické mapování paralelní strukturu origami,
což vede ke dvěma sadám ovládání, které využívají úhlové a lineární spouštěče.
Rozsah ohybu centrální pružiny a statické vlastnosti pružiny
jsou simulovány metodou konečných prvků. Simulované výsledky jsou 
ověřeny experimentálně a jsou použity při návrhu
nového ovládacího systému ze slitiny s tvarovou pamětí.

Při výpočtu a simulacích modelu origami byla adresována
možnost dalšího zmenšení modelu a velikosti ramena.
Se sestaveným fyzickým modelem drapáku byly vyhodnoceny uchopovací síly 
pomocí experimentálních testů.
S výrobním procesem inteligentní kompozitní mikrostruktury
byly vyrobeny prototypy modelu origami
pro ověření škálovatelnosti navrženého koncepčního návrhu.
Po vyhodnocení testů autoři dosáhli slibných výsledků
využití origami drapáku k úchytu předmětů.


%%%%%%%%%%%%%%%%%%%%%%%%%%%%%%%%%%%%%%%%%%%%%%%%%%%%%%%%%%%%
\section{Modely a strojové učení}
\label{sec:models}
Stejně jako je hardware a ovládací technika neustále posouvána kupředu
tak i software a možnosti výpočetních operací. Proto jsou v této kapitole
popsány další zajímavé možnosti chirurgické robotiky,
které se úplně nesoustřeďují na konstrukci robotů samotných jako
na metody použití robotů.

\subsection{Učení humanoidního robota demonstrací}
\label{subsec:teaching}
Tento článek \textit{Toward Teaching by Demonstration for Robot-Assisted Minimally Invasive Surgery} \cite{teaching_humanoid}
byl inspirován omezenými možnostmi manipulace při laparoskopické chirurgii
při kinematických omezeních v místě vstupu.

Osvojení manipulačních dovedností z otevřené operace poskytuje
flexibilnější přístup k orgánům a díky tomu by mohl chirurgický robot
pracovat velmi inteligentním způsobem.
Během výuky demonstrací
je možno přenést manipulační dovednosti člověka na humanoidního
robota pomocí aktivního učení během několika demonstrovaných zákroků.

Tato práce si klade za cíl přenést pohybové dovednosti z mnoha
lidských demonstrací v otevřené chirurgii na robotické manipulátory
v roboticky asistované MIS pomocí učení demonstrací.
Při provádění naučených dovedností pomocí robota pro MIS by však mělo
být respektováno kinematické omezení.
Autoři navrhují novou metodiku integrací technik
kognitivního učení a vyvinutých řídících technik, což umožňuje robotu
být vysoce inteligentní, aby se naučil dovednosti starších chirurgů
a v budoucnu mohl provádět naučené chirurgické operace
v semiautonomní chirurgii.
Nakonec byly provedeny experimenty k ověření účinnosti navržené strategie
a výsledky demonstrují schopnost systému přenést lidské manipulační
dovednosti na robota v MIS.

\subsection{Učící se robot pro řízení kamer}
\label{seubsec:learning}
Autoři článku \textit{A learning robot for cognitive camera control in minimally invasive surgery} \cite{self_learning}
demonstrují autonomního robota pro navádění kamer.
Robot se sám učí, je kontextově citlivý a použitelný pro MIS.
Většina chirurgických robotů dnes jsou telemanipulátory
bez autonomních schopností. Autonomní systémy byly vyvinuty
pro laparoskopické navádění kamerou, avšak dodržují jednoduchá
pravidla a nepřizpůsobují se chování konkrétním úkolům, postupům
nebo chirurgům.

V článku je prezentována metodika, která umožňuje robotům s různou
kinematikou vnímat své prostředí, interpretovat je na bázi znalostí
a provádět akce s vědomím kontextu. V rámci výcviku bylo
jedním chirurgem provedeno dvacet operací s naváděním kamery člověkem.
Následně bylo experimentálně vyhodnoceno kognitivní ovládání
robotické kamery.
Systém VIKY EP a robot KUKA LWR 4 byly trénovány na datech z
manuálního navádění kamery po dokončení zaučení chirurga.
Podruhé byla k výcviku LWR použita pouze data z VIKY EP
a následně data z trénování LWR byly použity na nové učení LWR.

Doba trvání každé operace se s narůstajícími zkušenostmi robota
zkracovala z $1704s \pm 244s$ na $1406s \pm 112s$ a $1197s$. 
Kvalita navádění kamery (dobrá/neutrální/špatná) se zlepšila 
z $38,6/53,4/7,9$ na $49,4/46,3/4,1\%$ a $56,2/41,0/2,8\%$.
Kognitivní kamerový robot tedy zlepšil svůj výkon díky zkušenostem
a položil tak základ pro novou generaci kognitivních chirurgických
robot, které se přizpůsobí potřebám chirurga.

\subsection{Nový inteligentní systém pro semi-automatické MIS} %15 - new intelligent laparoscopic system
\label{subsec:automa}
Inteligentní chirurgičtí roboti mají velký potenciál ulevit
chirurgům při únavných zákrocích. U MIS přidání inteligentních
metod řízení k řízení laparoskopie má velký potenciál.
Článek \textit{Development of a novel intelligent laparoscope system for semi-automatic minimally invasive surgery} \cite{intelligent_system}
se tedy věnuje metodám jak přenést část úkonů na robota,
aby se chirurg mohl plně soustředit na svou práci.
Autoři upravili Jakobiho matici nezávislou na hloubce tak, aby
byla vhodná pro vyjádření pohybového omezení insertního nástroje v laparoskopii.
Byla navržena metoda inteligentního a autonomního nastavení
zorného pole chirurga, která umožňuje sledovat a předvídat
trajektorii pohybu chirurgických nástrojů.

Výsledek experimentu ukazuje, že navržená metoda by mohla
realizovat sledování chirurgických nástrojů a nastavení zorného
pole chirurga při zákroku autonomně. V případě kluze lze
předvídat trajektorii pohybu chirurgických nástrojů.
Inteligentní laparoskopický systém by mohl zlepšit inteligentní
úroveň systému chirurgického robota. Vzhledem k tomu,
že chirurgovi je poskytnuta "třetí ruka" je navrhovaný systém
značným vylepšením poloutonomního systému chirurgického robota.


%%%%%%%%%%%%%%%%%%%%%%%%%%%%%%%%%%%%%%%%%%%%%%%%%%%%%%%%%%%
\section{Závěr}
\label{sec:zaver}
Tato zpráva slouží jako krátké nastínění roboticky asistované chirurgie.
V \ref{sec:robot} kapitole  je prezentováno současné využití robotů v MIS
a několik odborných návrhů z posledních let. V kapitole \ref{sec:models}
je pak zmíněno pár softwarových metod řešení či usnadnění chirurgických zákroků.

I přesto, že je dnes trh dominován firmou Intuitive Surgical Inc.,
vidím v robotické chirurgii velký potenciál, a to právě kvůli množství
nově vyvíjených technologií, jak v oblasti konstrukce robotů, tak
jejich řídících metod a vybavení. Se stále menšími součástkami a výkonnější
elektronikou je možné se už zamýšlet nad možností plně autonomního
"výsadkového" chirurgického robota.


%%%%%%%%%%%%%%%%%%%%%%%%%%%%%%%%%%%%%%%%%%%%%%%%%%%%%%%%%%%%%
% References
\newpage
\begingroup
\makeatletter
\renewcommand\section{\@startsection {section}{99}{\z@}%
                                   {-3.5ex \@plus -1ex \@minus -.2ex}%
                                   {4.5ex \@plus.2ex}%
                                   {\large\bfseries}}
\makeatother


\bibliography{references}{}
\bibliographystyle{unsrt}
\endgroup

\end{document}